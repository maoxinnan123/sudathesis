%---------------------------------------------------------------------------%
%->> Frontmatter
%---------------------------------------------------------------------------%
%-
%-> 生成封面
%-
\maketitle% 生成中文封面
\MAKETITLE% 生成英文封面
%-
%-> 作者声明
%-
\makedeclaration% 生成声明页
%-
%-> 中文摘要
%-
\intobmk\chapter*{摘\quad 要}% 显示在书签但不显示在目录
\setcounter{page}{1}% 开始页码
\pagenumbering{Roman}% 页码符号

本文是中国科学院大学学位论文模板ucasthesis的使用说明文档。主要内容为介绍\LaTeX{}文档类ucasthesis的用法,以及如何使用\LaTeX{}快速高效地撰写学位论文。

\keywords{中国科学院大学,学位论文,\LaTeX{}模板}% 中文关键词
%-
%-> 英文摘要
%-
\intobmk\chapter*{Abstract}% 显示在书签但不显示在目录

Two-dimensional transition metal dichalcogenides (TMDs) nanomaterials, such as MoS2 and WSe2, have proved to be promising electrocatalysts for CO2 reduction, in which the reduction product is CO with high selectivity. In order to improve the sluggish CO desorption process and the overall electrocatalytic performance, we have extensively explored the optimal dopants in MoS2 by high-throughput density functional theory (DFT) calculations and demonstrated the enhanced electrocatalytic activity. The dopants of V, Zr, and Hf into MoS2 could significantly promote the desorption of CO from the MoS2 edge, achieving the optimal performance for electrocatalytic CO2 reduction. Furthermore, the dopants locating close to the active Mo site is crucial to influence the catalytic activity, while the dopant concentration is not important. The modulating strategy we proposed here also applies to other TMDs materials for enhancing electrocatalytic activity.

\KEYWORDS{Impurities, Redox reactions, Binding energy, Adsorption, Doping}% 英文关键词
%---------------------------------------------------------------------------%
